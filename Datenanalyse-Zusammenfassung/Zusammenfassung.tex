\documentclass{article}
\usepackage[formats]{listings}
\usepackage[utf8]{inputenc}
\usepackage{xcolor}
\usepackage{amssymb}

\title{Datenanalyse}

\begin{document}
	\maketitle
	\section{Basics}
	Mittelwert (diskret/stetig)
	\[\bar{x} = \sum_{i=1}^{n}(x_i) \]	
	\[\mathbb{E}[X]=\int_{-\infty}^{\infty} (f(x)*x) dx\]
	Varianz (diskret/stetig)
	\[{\sigma}^{2} = \frac{1}{n}*\sum_{i=1}^{n}(x_i - \bar{x})^2\]
	\[{\sigma}^{2} = \mathbb{E}(X^2)-(\mathbb{E}(X))^2\]
	Geschätzte Varianz
	\[{s}^{2} = \frac{1}{n-1}*\sum_{i=1}^{n}(x_i - \bar{x})^2\]
	Pearson Korrelation
	\[r=\frac{\sum(x_i-\bar{x})(y_i-\bar{y})}{(N-1)s_x s_y}\]
	\section{Faltung}
	\subsection{Diskret}
	\[f_Z(z)=\sum_{i=0}^{\infty}\sum_{j=0}^{\infty}f_X(x_i)*f_Y(x_j)*\mathbb{I}_{z}(x_i+y_j)\]
	Wenn beide Träger teil der natürlichen Zahlen sind, dann darf folgende Formel benutzt werden. ($T_1=T_2=\mathbb{N}_0$)
	\[f_Z(z)=\sum_{i=0}^{z}f_X(i)*f_Y(z-i)\]
	\subsection{Stetig}
	Wenn X und Y stetig verteilte Zufallsvariablen sind und $Z=X+Y$. Dann gilt
	\[f_Z(z)=f_{X+Y}(z)=\int_{-\infty}^{\infty}f_X(x)f_Y(z-x)dx=\int_{-\infty}^{\infty}f_X(z-y)f_Y(y)dy\]
	\subsection{Bivariate}	
	\[(f*g)(x):= \int f(T) * g(x - T) dT\]
	http://www.biancahoegel.de/mathe/analysis/faltung.html
	\subsection{Dichtetransformationssatz}
	Es ist eine Funktion ''f'' gegeben und eine Funktion ''g''
	\[h = {g}^{-1}\]
	\[f = |h'| * f(h) * \mathbb{I}_\Omega \ \]
	\section{Schätzer}
	To do...
	\begin{itemize}
		\item MSE
		\item Konsistenz von Schätzern
	\end{itemize}
	\subsection{Maximum-Likelihood-Methode}
	Likelihood Funktion
	\[L = f(x1) * ... * f(xn)\]
	Einsetzten der Likelihood Funktion in den Logarithmus.
	\[l(p) = log(L) = log(f_{x1})+...+log(f_{xn}) \]
	Erste Ableitung bilden
	\[l(p)' = \frac{\delta}{\delta p} l(p)\]
	Zweite Ableitung bilden
	\[l(p)'' = \frac{\delta}{\delta p} l'(p)\]
	Notwendige Bedingung
	\[l(p)' = 0\]
	Hinreichende Bedingung
	\[l(p)' \ne 0 \]
	\subsection{Momentenmethode}
	Erwartungswert berechnen und mit dem Empirischen Wert gleichsetzten
	\[\mathbb{E}[X]=\bar{x}\]
	\subsection{Gütekriterien für Punktschätzer}
	Erwartungstreue ist, wenn für einen Schätzer gilt, dass der Erwartungswert ($E_\theta(T)$) gleich dem geschätzten Wert ($\theta$) ist. Also:
	\[E_\theta(T)=\theta\] 
	Der Bias beschreibt, wie viel der Schätzer vom Erwartungswert abweicht.
	\[Bias_\theta(T)=E_\theta-\theta\]
	Zudem gibt es noch asymptotische Erwartungstreue. Das bedeutet, dass bei unendlich viele Beobachtungen, der Schätzer erwartungstreu wird ($n -> \infty$)
	\[\lim\limits_{n -> \infty} E_\theta(T) = \theta\]
	\section{Konfidenz Intervall}
	Erwartungswert
	\[[\bar{x}-z_{1-\alpha/2}*\frac{\sigma}{\sqrt{n}};\bar{x}+z_{1-\alpha/2}*\frac{\sigma}{\sqrt{n}}]\]
	Varianz
	\[[\frac{(n-1*S^2)}{q_{1-a/2}};\frac{(n-1*S^2)}{q_{a/2}}]\]
	\section{Hypothesen Tests}
	To do...
	\begin{itemize}
		\item ab 6.4 unvollständig 
	\end{itemize}
	\[v=\sqrt{n}*\frac{\bar{x}-h_0}{\sigma}\]
	Einstichprobentest (Bernoulli Experiment)
	\[v=\sqrt{n}*\frac{\bar{x}-p_0}{\sqrt{q*(1-q)}}\]
	\linebreak 
	\[p=2*(1-\phi(|v|))\]
	phi ist eine Verteilungsfunktion.
	\linebreak
	Entscheidungsregeln (bekanntes $\sigma$):
	\begin{itemize}
		\item $H_0 : \mu = \mu_0$ vs $H_1 : \mu \ne \mu_0$ : $H_0$ wird abgelehnt, falls $|v| > z_{1-\alpha/2}$
		\item $H_0 : \mu \ge \mu_0$ vs $H_1 : \mu < \mu_0$ : $H_0$ wird abgelehnt, falls $v < z_{\alpha}$
		\item $H_0 : \mu \le \mu_0$ vs $H_1 : \mu > \mu_0$ : $H_0$ wird abgelehnt, falls $v > z_{1-\alpha}$
	\end{itemize}
	Entscheidungsregeln (unbekanntes $\sigma$):
	\begin{itemize}
		\item $H_0 : \mu = \mu_0$ vs $H_1 : \mu \ne \mu_0$ : $H_0$ wird abgelehnt, falls $|v| > t_{1-\alpha/2}$
		\item $H_0 : \mu \ge \mu_0$ vs $H_1 : \mu < \mu_0$ : $H_0$ wird abgelehnt, falls $v < t_{\alpha}$
		\item $H_0 : \mu \le \mu_0$ vs $H_1 : \mu > \mu_0$ : $H_0$ wird abgelehnt, falls $v > t_{1-\alpha}$
	\end{itemize}
	
	\section{Regression}
	\subsection{Einfache lineare Regression}
	\[f(x)=a+b*x\]
	\[b = \frac{\sum_{i=1}^{n}(x_i y_i)-n\bar{x}\bar{y}}{\sum_{i=1}^{n}(y^2)-n\bar{y}^2}\]
	\[a=\bar{y}-b*\bar{x}\]
	Es ist hilfreich eine Tabelle mit folgenden Spalten zu erstellen:
	\begin{center}
		\begin{tabular}{|| c c c c c ||}
			\hline
			$x$ & $y$ & $x^2$ & $y^2$ & $xy$ \\ 
			\hline
			\hline
			$\sum x$ & $\sum y$ & $\sum x^2$ & $\sum y^2$ & $\sum xy$ \\ \hline
			
		\end{tabular}
	\end{center}
	Zudem sollten auch $\bar{x}$ und $\bar{y}$ ausgerechnet werden
	
	\subsection{Multiple Lineare Regression}
	\[\beta = (X^T*X)^{-1}*X^T*Y\] falls $(X^T*X)^{-1}$ existiert

\end{document}



